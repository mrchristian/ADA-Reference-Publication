\documentclass{article}

\usepackage{hyperref}
\begin{document}

\title{On Combinations of Masters Against the Public}

\maketitle


Source: \textbf{On the economy of machinery and manufactures, Charles Babbage, 1846}


Archive.org \href{https://archive.org/details/oneconomyofmachi00babbrich/page/312/mode/2up}{https://archive.org/details/oneconomyofmachi00babbrich/page/312/mode/2up} 


(376.) A SPECIES of combination occasionally takes place amongst manufacturers against persons having patents : and these combinations are always injurious to the public, as well as unjust to the inventors. Some years since, a gentleman invented a machine, by which modellings and carvings were cut in mahogany, and other fine woods. The machine resembled, in some measure, the drilling apparatus employed in ornamental lathes ; it produced beautiful work at a very moderate expense : but the cabinet-makers met together, and combined against it, and the patent has consequently never been worked. A similar fate awaited a machine for cutting veneers by means of a species of knife. In this instance, the wood could be cut thinner than by the circular saw, and no waste was incurred ; but "the trade" set themselves against it, and after a heavy expense, it was given up.


The excuse alleged for this kind of combination, was the fear entertained by the cabinet-makers that when the public became acquainted with the article, the patentee would raise the price.


Similar examples of combination seem not to be unfrequent, as appears by the Report of the Committee of the House of Commons on Patents for Inventions, June, 1829. See the evidence of Mr, Holds worth.


(377.) There occurs another kind of combination against the public, with which it is difficult to deal. It usually ends in a monopoly, and the public are then left to the discretion of the monopolists not to charge them ahove the "growling point;" that is, not to make them pay so much as to induce them actually to combine against the imposition. This occurs when two companies supply water or gas to consumers by means of pipes laid down under the pavement in the streets of cities : it may possibly occur also in docks, canals, rail-roads, \&c., and in other cases where the capital required is very large, and the competition very limited. If water or gas companies combine, the public immediately loses all the advantage of competition, and it has generally happened, that at the end of a period during which they have undersold each other, the several companies have agreed to divide the whole district supplied, into two or more parts, each company then removing its pipes from all the streets except those in its own portion. This removal causes great injury to the pavement, and when the pressure of increased rates induces a new company to start, the same inconvenience is again produced. Perhaps one remedy against evils of this kind might be, when a charter is granted to such companies, to restrict, to a certain amount, the rate of profit on the shares, and to direct that any profits beyond, shall accumulate for the repayment of the original capital. This has been done in several late acts of parliament establishing companies. The maximum rate of profit allowed ought to be liberal, to compensate for the risk ; the public ought to have auditors on their part, and the accounts should be annually published, for the purpose of preventing the limitations from being exceeded. It must however be admitted, that this would be an interference with capital, which, if allowed, should, in the present state of our knowledge, be examined with great circumspection in each individual case, until some general principle is established on well-admitted grounds.


(378.) An instrument called a gas-meter, which ascertains the quantity of gas used by each consumer, has been introduced, and furnishes a satisfactory mode of determining the payments to be made by individuals to the Gas companies. A contrivance somewhat similar in its nature, might be used for the sale of water ; but in that case some public inconvenience might be apprehended, from the diminished quantity which would then run to waste : the streams of water running through the sewers in London, are largely supplied from this source ; and if this supply were diminished, the drainage of the metropolis might be injuriously affected.


(379.) In the north of England a powerful combination has long existed among the coal-owners, by which the public has suffered in the payment of increased price. The late examination of evidence before a Committee of the House of Commons, has explained its mode of operation, and the Committee have recommended, that for the present the sale of coal should be left to the competition of other districts.


(380.) A combination, of another kind, exists at this moment to a great extent, and operates upon the price of the very pages which are now communicating information respecting it. A subject so interesting to every reader, and still more so to every manufacturer of the article which the reader consumes, deserves an attentive examination.


We have shown in Chap. XXI. p. 205, the component parts of the expense of each copy of the present work ; and we have seen that the total amount of the cost of its production, exclusive of any payment to the author for his labour, is 2s. 3d.*


Another fact, with which the reader is more practically familiar, is, that he has paid, or is to pay, to his bookseller, six shillings for the volume. Let us now examine into the distribution of these six shillings, and then, having the facts of the case before us, we shall be better able to judge of the merits of the combination just mentioned, and to explain its effects.


Distribution of the Profits on a Six Shilling Book.




BUYS AT


SELLS AT


PROFIT on Capital expended.


No. I. The Publisher who ac\} counts to the author for > every copy received . . j


.. d.


3 10


*. d. 4 2


10 per cent.


No. 1 1. The Bookseller who re") tails to the public . )


Or .


4 2 4 6


6 6


44 334


No. I. the Publisher, is a bookseller ; he is, in fact, the Author's agent. His duties are, to receive and take charge of the stock, for which he supplies warehouse-room ; to advise the author about the times and methods of advertising ; and to insert the advertisements. As he publishes other books, he


* The whole of the subsequent details relate to the first edition of this work.


will advertise lists of those sold by himself; and thus, by combining many in one advertisement, diminish the expense to each of his principals. He pays the author only for the books actually sold ; consequently, he makes no outlay of capital, except that which he


ays for advertisements : but he is answerable for -iny bad debts he may contract in disposing of them. His charge is usually ten per cent, on the returns.


No. II. is the Bookseller who retails the work to the public. On the publication of a new book, the Publisher sends round to the trade, to receive " subscriptions " from them for any number of copies not less than two. These copies are usually charged to the " subscribers," on an average, at about four or five per cent, less than the wholesale price of the book : in the present case the subscription price is 4s. 2d. for each copy. After the day of publication, the price charged by the publisher to the booksellers is 4s. 6d. With some works it is the custom to deliver twenty-five copies to those who order twentyfour, thus allowing a reduction of about four per cent. Such was the case with the present volume. Different publishers offer different terms to the subscribers ; and it is usual, after intervals of about six months, for the publisher again to open a subscription list, so that if the work be one for which there is a steady sale, the trade avail themselves of these opportunities of purchasing, at the reduced rate, enough to supply their probable demand. *


(381.) The volume thus purchased of the pub-


* These details vary with different books and different publishers; those given in the text are believed to be substantially correct, and are applicable to works like the present.


lisher at 4s. 2df. or 4s. 6d. is retailed by the book-* seller to the public at 6s. In the first case he makes a profit of forty-four, in the second of thirty-three per cent. Even the smaller of these two rates of profit on the capital employed, appears to be much too large. It may sometimes happen, that when a book is inquired for, the retail dealer sends across the street to the wholesale agent, and receives, for this trifling service, one fourth part of the money paid by the purchaser ; and perhaps the retail dealer takes also six months' credit for the price which the volume actually cost him.


(382.) In section 256, the price of each process in manufacturing the present volume was stated : we shall now give an analysis of the whole expense of conveying it into the hands of the public.


s. d. The retail price 6s. on 3052 produces 915 12


1 Total expense of printing and paper . . 207 5 8 T ?j.


2 Taxes on paper and advertisements .... 40 Oil


3 Commission to publisher as


agent between author and


printer 1814 4 T * T


4 Commission to publisher as


agent for sale of the book . 6311 8


5 Profit : tbe difference be-


tween subscription price


and trade price, 4d. per vol. 50 17 4


6 Profit : the difference be-


tween trade price and retail


price, Is. 6d. per vol 228 18


* 362 1 4\_ T


7 Remains for authorship , . . . 306 4


Total . 915 12


This account appears to disagree with that in page 206 ; but it will be observed that the three first articles amount to 2661. Is., the sum there stated. The apparent difference arises from a circumstance which was not noticed in the first edition of this work. The bill amounting to 205 . 18s., as there given, and as reprinted in the present volume, included an additional charge of ten per cent, upon the real charges of the printer and paper-maker.


(383.) It is usual for the publisher, when he is employed as agent


between the author and printer, to charge a commission of ten per cent, on all payments he makes. If the author is informed of this custom previously to his commencing the work, as was the case in the present instance, he can have no just cause of complaint ; for it is optional whether he himself employs the printer, or communicates with him through the intervention of his publisher.


The services rendered for this payment are, the making arrangements with the printer, the woodcutter, and the engraver, if required. There is a convenience in having some intermediate person between the author and printer, in case the former should consider any of the charges made by the latter as too high. When the author himself is quite unacquainted with the details of the art of printing, he may object to charges which, on a better acquaintance with the subject, he might be convinced were very moderate and in such cases he ought to depend on the judgment of his publisher, who is generally conversant with the art. This is particularly the case in the charge for alterations and corrections, some of which, although apparently trivial, occupy




ffrr rr T




the compositors much time in ma


also be observed that the publisher, in


comes responsible for the payments to those persons.


(384.) It is not necessary that the author should avail himself of this intervention, although it is the interest of the publisher that he should ; and booksellers usually maintain that the author cannot procure his paper or printing at a cheaper rate if he go at once to the producers. This appears from the evidence given before the Committee of the House of Commons in the Copyright Acts, May 8, 1818.


Mr. O. Rees, bookseller, of the house of Longman and Co., Paternoster-row, examined :


" Q. Suppose a gentleman to publish a work on his own account, and to incur all the various expenses ; could he get the paper at 30s. a ream ?


" A. I presume not ; I presume a stationer would not sell the paper at the same price to an indifferent gentleman as to the trade.


" Q. The Committee asked you if a private gentleman was to publish a work on his own account, if he would not pay more for the paper than persons in the trade ; the Committee wish to be informed whether a printer does not charge a gentleman a higher rate than to a publisher.


" A. I conceive they generally charge a profit on the paper.


" Q. Do not the printers charge a higher price also for printing, than they do to the trade ?


" A. I always understood that they do."


(385.) There appears to be little reason for this distinction in charging for printing a larger price to the author than to the publisher, provided the former is able to give equal security for the payment. With respect to the additional charge on paper, if the


author employs either publisher or printer to purchase it, they ought to receive a moderate remuneration for the risk, since they become responsible for the payment; but there is no reason why, if the author deals at once with the paper-maker, he should not purchase on the same terms as the printer ; and if he choose, by paying ready money, not to avail himself of the long credit allowed in those trades, he ought to procure his paper considerably cheaper.


(386.) It is time, however, that such conventional combinations between different trades should be done away with. In a country so eminently depending for its wealth on its manufacturing industry, it is of importance that there should exist no abrupt distinction of classes, and that the highest of the aristocracy should feel proud of being connected, either personally or through their relatives, with those pursuits on which their country's greatness depends. The wealthier manufacturers and merchants already mix with those classes, and the larger and even the middling tradesmen are frequently found associating with the gentry of the land. It is good that this ambition should be cultivated, not by any rivalry in expense, but by a rivalry in knowledge and in liberal feelings ; and few things would more contribute to so desirable an effect, than the abolition of all such contracted views as those to which we have alluded. The advantage to the other classes, would be an increased acquaintance with the productive arts of the country, an increased attention to the importance of acquiring habits of punctuality and of business, and, above all, a general feeling that it is honourable, in any rank of life, to increase our own and our country's riches, by


employing our talents in the production or in the distribution of wealth.


(387.) Another circumstance omitted to be noticed in the first edition relates to what is technically called " the overplus" which may be now explained. When 500 copies of a work are to be printed, each sheet of it requires one ream of paper. Now a ream, as used by printers, consists of 21 J quires, or 516 sheets. This excess of sixteen sheets is necessary in order to allow for " revises," for preparing and adjusting the press for the due performance of its work, and to supply the place of any sheets which may be accidentally dirtied or destroyed in the processes 01 printing, or injured by the binder in putting into boards. It is found, however, that three per cent, is more than the proportion destroyed, and that damage is less frequent in proportion to the skill and care of the workmen.


From the evidence of several highly respectable booksellers and printers, before the Committee of the House of Commons on the Copyright Act, May, 1818, it appears that the average number of surplus copies, above 500, is between two and three ; that on smaller impressions it is less, whilst on larger editions it is greater ; that, in some instances, the complete number of 500 is not made up, in which case the printer is obliged to pay for completing it ; and that in no instance have the whole sixteen extra copies been completed. On the volume in the reader's hands, the edition of which consisted of 3000, the surplus amounted to fifty-two, a circumstance arising from the improvements in printing and the increased care of the pressmen. Now this overplus ought to be accounted for to the author; and I believe it usually is so by all respectable publishers.


(388.) In order to prevent the printer from privately taking off a larger number of impressions than he delivers to the author or publisher, various expedients have been adopted. In some works a particular water-mark has been used in paper made purposely for the book : thus the words " Mecanique Coeleste " appear in the water-mark of the two first volumes of the great work of Laplace. In other cases, where the work is illustrated by engravings, such a fraud would be useless without the concurrence of the copper-plate printer. In France it is usual to print a notice on the back of the title-page, that no copies are genuine without the subjoined signature of the author: and attached to this notice is the author's name, either written, or printed by hand from a wooden block. But notwithstanding this precaution, I have recently purchased a volume, printed at Paris, in which the notice exists, but no signature is attached. In London there is not much danger of such frauds, because the printers are men of capital, to whom the profit on such a transaction would be trifling, and the risk of the detection of a fact, which must of necessity be known to many of their workmen, would be so great as to render the attempt at it folly.


(389.) Perhaps the best advice to an author, if he publishes on his own account, and is a reasonable person, possessed of common sense, would be to go at once to a respectable printer and make his arrangements with him.


(390.) If the author do nM wish to print his work at his own risk, then he should make an agreement with a publisher for an edition of a limited number ; but he should by no means sell the copyright. If the work contains wood-cuts or engravings, it would be judicious to make it part of the contract that they shall become the author's property, with the view to their use in a subsequent edition of the works, if they should be required. An agreement is frequently made by which the publisher advances the money and incurs all the risk on condition of his sharing the profits with the author. The profits alluded to are, for the present work, the last item of section 382, or, 306/. 4s.


(391.) Having now explained all the arrangements in printing the present volume, let us return to section 382, and examine the distribution of the 915Z. paid by the public. Of this sum 207Z. was the cost of the book, 40/. was taxes, 3621. was the charges of the bookseller in conveying it to the consumer, and 306Z. remained for authorship.


The largest portion, or 362L goes into the pockets of the booksellers; and as they do not advance capital, and incur very little risk, this certainly appears to be an unreasonable allowance. The most extravagant part of the charge is the thirty-three per cent, which is allowed as profit on retailing the book.


It is stated, however, that all retail booksellers allow to their customers a discount of ten per cent, upon orders above 205., and that consequently the nominal profit of forty-four or thirty-three per cent, is very much reduced. If this is the case, it may fairly be inquired, why the price of 2J. for example, is printed upon the back of a book, when every bookseller is ready to sell it at 11. 16s., and why those who are unacquainted with that circumstance should be made to pay more than others who are better informed ?


(392.) Several reasons have been alleged as justifying this high rate of profit.


1st. It has been alleged that the purchasers of books take long credit. This, probably, is often the case, and admitting it, no reasonable person can object to a proportionate increase of price. But it is no less clear, that persons who do pay ready money, should not be charged the same price as those who defer their payments to a remote period.


2d. It has been urged that large profits are necessary to pay for the great expenses of bookselling establishments ; that rents are high and taxes heavy ; and that it would be impossible for the great booksellers to compete with the smaller ones, unless the retail profits were great. In reply to this it may be observed that the booksellers are subject to no peculiar pressure which does not attach to all other retail trades. It may also be remarked that large establishments always have advantages over smaller ones, in the economy arising from the division of labour ; and it is scarcely to be presumed that booksellers are the only class who, in large concerns, neglect to avail themselves of them.


3d. It has been pretended that this high rate of profit is necessary to cover the risk of the bookseller's having some copies left on his shelves ; but he is not obliged to buy of the publisher a single copy more than he has orders for : and if he do purchase more, at the subscription price, he proves, by the very fact, that he himself does not estimate that risk at more than from four to eight per cent.


(393.) It has been truly observed, on the other hand, that many copies of books are spoiled by persons who enter the shops of booksellers without intending to make any purchase. But, not to mention that such persons finding on the tables various new publications, are frequently induced, by that opportunity of inspecting them, to become purchasers : this damage does not apply to all booksellers nor to all books ; of course it is not necessary to keep in the shop books of small probable demand or great price. In the present case, the retail profit on three copies only, namely, 4s. 6d., would pay the whole cost of the one copy soiled in the shop ; and even that copy might afterwards produce, at an auction, half or a third of its cost price. The argument, therefore, from disappointments in the sale of books, and that arising from heavy stock, are totally groundless in the question between publisher and author. It should be remarked also, that the publisher is generally a retail, as well as a wholesale, bookseller ; and that, besides his profit upon every copy which he sells in his capacity of agent, he is allowed to charge the author as if every copy had been subscribed for at 4s. 2d., and of course he receives the same profit as the rest of the wholesale traders for the books retailed in his own shop.


(394.) In the country, there is more reason for a considerable allowance between the retail dealer and the public ; because the profit of the country bookseller is diminished by the expense of the carriage of the books from London. He must also pay a commission, usually five per cent., to his London agent, on all those books which his correspondent does not himself publish. If to this be added a discount of five per cent., allowed for ready money to every customer, and of ten per cent, to book-clubs, the profit of the bookseller in a small country town is by no means too large.


Some of the writers, who have published criticisms on the observations made in the first edition of this work, have admitted that the apparent rate of profit to the booksellers is too large. But they have, on the other hand, urged that too favourable a case is taken in supposing the whole 3000 copies sold. If the reader will turn back to section 382, he will find that the expense of the three first items remains the same, whatever be the number of copies sold ; and on looking over the remaining items he will perceive that the bookseller, who incurs very little risk and no outlay, derives exactly the same profit per cent, on the copies sold, whatever their number may be. This, however, is not the case with the unfortunate author, on whom nearly the whole of the loss falls undivided. The same writers have also maintained, that the profit is fixed at the rate mentioned, in order to enable the bookseller to sustain losses, unavoidably incurred in the purchase and retail of other books. This is the weakest of all arguments* It would be equally just that a merchant should charge an extravagant commission for an undertaking unaccompanied with any risk, in order to repay himself for the losses which his own want of skill might lead to in his other mercantile transactions.


(395.) That the profit in retailing books is really too large, is proved by several circumstances : First, That the same nominal rate of profit has existed in the bookselling trade for a long series of years, notwithstanding the great fluctuations in the rate of profit on capital invested in every other business. Secondly, That, until very lately, a multitude of booksellers, in all parts of London, were content with a much smaller profit, and were willing to sell for ready money, or at short credit, to persons of undoubted character, at a profit of only ten per cent., and in some instances even at a still smaller percentage, instead of that of twenty-five per cent, on the published prices. Thirdly, that they are unable to maintain this rate of profit except by a combination, the object of which is to put down all competition.


(396.) Some time ago a small number of the large London booksellers entered into such a combination. One of their objects was to prevent any bookseller from . \ selling books for less than ten per cent, under the published prices ; and in order to enforce this principle, they refuse to sell books, except at the publishing price, to any bookseller who declines signing an agreement to that effect. By degrees, many were prevailed upon to join this combination; and the effect of the exclusion it inflicted, left the small capitalist no option between signing or having his business destroyed. Ultimately, nearly the whole trade, comprising about two thousand four hundred persons, have been compelled to sign the agreement. As might be naturally expected from a compact so injurious to many of the parties to it, disputes have arisen ; several booksellers have been placed under the ban of the combination, who allege that they have not violated its rales, and who accuse the opposite party of using spies, \&c. to entrap them.*


(397.) The origin of this combination has been explained by Mr. Pickering, of Chancery-lane, himself a publisher, in a printed statement, entitled, "BOOKSELLERS' MONOPOLY ; " and the following list of booksellers, who form the committee for conducting this combination, is copied from that printed at the head of each of the cases published by Mr. Pickering :


" Allen, J., 7, Leadenhall-street.


" Arch, J., 61, Cornhill.


" Baldwin, R., 47, Paternoster-row.


" Booth, J.


" Duncan, J., 37, Paternoster-row.


" Hatchard, J., Piccadilly.


" Marshall, R., Stationers '-court.


" Murray, J., Albemarle-street.


" Rees, O., 39, Paternoster-row.


" Richardson, J. M., 23, Cornhill.


" Rivington, J., St. Paul's Church-yard.


" Wilson, E., Royal Exchange." (398.) In whatever manner the profits are divided oetween the publisher and the retail bookseller, the fact remains, that the reader pays for the volume in his hands 6s., and that the author will receive only 3s. Wd. ; out of which latter sum, the expense of printing the volume must be paid : so that in passing


* It is now understood that the use of spies has been given up; and it is also known that the system of underselling is again privately resorted to by many ; so that the injury arising from this arbitrary system, pursued by the great booksellers, affects only, or most severely, those whose adherence to an extorted promise most deserves respect. Note to the second edition.


through two hands this book has produced a profit of forty-four per cent. This excessive rate of profit has drawn into the book-trade a larger share of capital than was really advantageous ; and the competition between the different portions of that capital has naturally led to the system of underselling, to which the committee above-mentioned are endeavouring to put a stop. *


(399.) There are two parties who chiefly suffer from this combination., the public and authors. The first party can seldom be induced to take an active part against any grievance ; and in fact little is required from it, except a cordial support of the authors, in any attempt to destroy a combination so injurious to the interests of both.


Many an industrious bookseller would be glad to sell for 5s. the volume which the reader holds in hrs hand, and for which he has paid 65. ; and, in doing so for ready money, the tradesman who paid 4s. 6d. for the book, would realise, without the least risk, a profit of eleven per cent, on the money he had advanced. It is one of the objects of the combination we are discussing, to prevent the small capitalist from employing his capital at that rate of profit which he thinks most advantageous to himself; and such a proceeding is decidedly injurious to the public.


(400.) Having derived little pecuniary advantage


* The Monopoly Cases, Nos. 1, 2, and 3, of those published by Mr. Pickering, should be consulted upon this point ; and, as the public will be better able to form a judgment by hearing the other side of the question, it is to be hoped the Chairman of the Committee (Mr. Richardson) will publish those Regulations respecting the trade, a copy of which, Mr. Pickering states, is refused


by the Committee even to those who sign them. 


from my own literary productions ; and being aware, that from the very nature of their subjects, they can scarcely be expected to reimburse the expense of preparing them, I may be permitted to offer an opinion upon the subject, which I believe to be as little influenced by any expectation of advantage from the future, as it is by any disappointment at the past. Before, however, we proceed to sketch the plan of a campaign against Paternoster-row, it will be fit to inform the reader of the nature of the enemies' forces, and of his means of attack and defence. Several of the great publishers find it convenient to be the proprietors of Reviews, Magazines, Journals, and even of Newspapers. The Editors are paid, in some instances very handsomely, for their superintendence ; and it is scarcely to be expected that they should always mete out the severest justice on works by the sale of which their employers are enriched. The great and popular works of the day are, of course, reviewed with some care, and with deference to public opinion. Without this, the journals would not sell ; and it is convenient to be able to quote such articles as instances of impartiality. Under shelter of this, a host of ephemeral productions are written into a transitory popularity ; and by the aid of this process, the shelves of the booksellers, as well as the pockets of the public, are disencumbered. To such an extent are these means employed, that some of the periodical publications of the day ought to be regarded merely as advertising machines. That the reader may be in some measure on his guard against such modes of influencing his judgment, he should examine whether the work reviewed is published bv the bookseller who is the proprietor of the review ; a fact which can sometimes be ascertained from the title of the book as given at the head of the article. But this is by no means a certain criterion, because partnerships in various publications exist between houses in the book trade, which are not generally known to the public ; so that, in fact, until Reviews are establisned in which booksellers have no interest, they can never be safely trusted.


(401.) In order to put down the combination of booksellers, no plan appears so likely to succeed as a counter-association of authors. If any considerable portion of the literary world were to unite and form such an association ; and if its affairs were directed by an active committee, much might be accomplished. The objects of such an union should be, to employ some person well skilled in the printing, and in the bookselling trade ; and to establish him in some central situation as their agent. Each member of the association to be at liberty to place any, or all of his works in the hands of this agent for sale ; to allow any advertisements, or list of books published by members of the association, to be stitched up at the end of each of his own productions ; the expense of preparing them being defrayed by the proprietors of the books advertised.


The duties of the agent would be to retail to the public, for ready money, copies of books published by members of the association. To sell to the trade, at prices agreed upon, any copies they may require. To cause to be inserted in the journals, or at the end of works published by members, any advertisements which the committee or authors may direct. To prepare a general catalogue of the works of members. To be the agent for any member of the association respecting the printing of any work. Such a union would naturally present other advantages ; and as each author would retain the liberty of nutting any price he might think fit on his producdons, the public would have the advantage of reduction in price produced by competition between authors on the same subject, as well as of that arising from a cheaper mode of publishing the volumes sold to them.


(402.) Possibly, one of the consequences resulting from such an association, would be the establishment of a good and an impartial Review, a work the want of which has been felt for several years. The two long-established and celebrated Reviews, the unbending champions of the most opposite political opinions, are, from widely differing causes, exhibiting unequivocal signs of decrepitude and decay. The Quarterly advocate of despotic principles is fast receding from the advancing intelligence of the age ; the new strength and new position which that intelligence has acquired, demands for its expression, new organs, equally the representatives of its intellectual power, and of its moral energies : whilst, on the other hand, the sceptre of the Northern critics has passed, from the vigorous grasp of those who established its dominion, into feebler hands.


(403.) It may be stated as a difficulty in realizing this suggestion, that those most competent to supply periodical criticism, are already engaged. But it is to be observed, that there are many who now supply literary criticisms to journals, the political principles of which they disapprove ; and that if once a respectable and well-supported Review* were established, capable of competing, in payment to its contributors, with the wealthiest of its rivals, it would very soon be supplied with the best materials the country can produce, f It may also be apprehended that such a combination of authors would be favourable to each other. There are two temptations to which an Editor of a review is commonly exposed : the first is, a tendency to consult too much, in the works he criticises, the interest of the proprietor of his review ; the second, a similar inclination to consult the interests of his friends. The plan which has been proposed removes one of these temptations, but it would be very difficult, if not impossible, to destroy the other.


* At the moment when this opinion as to the necessity for a new Review was passing through the press, I was informed that the elements of such an undertaking were already organized.


f It has been suggested to me, that the doctrines maintained in this chapter may subject the present volume to the opposition of that combination which it has opposed. I do not entertain that opinion ; and for this reason, that the booksellers are too shrewd a class to supply such an admirable passport to publicity as their opposition would prove to be if generally suspected.* But should my readers take a different view of the question, they can easily assist in remedying the evil, by each mentioning the existence of this little volume to two of his friends.


* I was mistaken in this conjecture; all booksellers are not so shrewd as I had imagined, for some did refuse to sell this volume i consequently others sold a larger number of copies.


In the Preface to the second edition, at the commencement of this volume, the reader will find some farther observations on the effect of the Booksellers' combination. Note to the Second Edition.

\end{document}
