\documentclass{article}

\usepackage{hyperref}
\begin{document}

\title{Preface}

\maketitle


Charles Babbage June 8, 1832. \href{https://www.openstreetmap.org/#map=19/51.51673/-0.15303}{Dorset Street, Manchester Square, London}.


THE present volume may be considered as one of the consequences that have resulted from the CalculatingEngine, the construction of which I have been so long superintending. Having been induced, during the last ten years, to visit a considerable number of workshops and factories, ootfi in England and on the Continent, for the purpose of endeavouring to rcake myself acquainted with the various resources of mechanical art, I was insensibly led to apply to them those principles of generalization to which my other pursuits had naturally given rise. The increased number of curious processes and interesting facts which thus came under my attention, as well as of the reflections which they suggested, induced me to believe that the publication of some of them might be of use to persons who propose to bestow their attention on those inquiries which I have only incidentally considered. With this view it was my intention to have delivered the present work in the form of a course of lectures at Cambridge; an intention which I was subsequently induced to alter. The substance of a considerable portion of it has, however, appeared among the preliminary chapters of the mechanical part of the Encyclopaedia Metropolitana.


I have not attempted to offer a complete enumeration of all the mechanical principles which regulate the application ot machinery to arts and manufactures, but I have endeavoured to present to the reader those which struck me as the most important, either for understanding the actions of machines, or for enabling the memory to classify and arrange the facts connected with their employment. Still less have I attempted to examine all the difficult questions of political economy which are intimately connected with such inquiries. It was impossible not to trace or to imagine, among the wide variety of facts presented m me, some principles which seemed to pervade many establishments; and having formed such conjectures, the desire to refute or to verify them, gave an additional interest to the pursuit. Several of the principles which I have proposed, appear to me to have been unnoticed before. This was particularly the case with respect to the explanation I have given of the division of labour ; but further inquiry satisfied me that I had been anticipated by M. Gioja, and it is probable that additional research would enable me to trace most of the other principles, which I had thought original, to previous writers, to whose merit I may perhaps be unjust, from my want of acquaintance with the historical branch of the subject.


The truth however of the principles I have stated, is of much more importance than their origin ; and the utility of an inquiry into them, and of establishing others more correct, if these should be erroneous, can scarcely admit of a doubt.


The difficulty of understanding the processes of manufactures has unfortunately been greatly overrated. To examine them with the eye of a manufacturer, so as to be able to direct others to repeat them, does undoubtedly require much skill and previous acquaintance with the subject ; but merely to apprehend their general principles and mutual relations, is within the power of almost every person possessing a tolerable education.


Those who possess rank in a manufacturing country, can scarcely be excused if they are entirely ignorant of principles, whose development has produced its greatness. The possessors of wealth can scarcely be indifferent to processes which, nearly or remotely, have been the fertile source of their possessions. Those who enjoy leisure can scarcely find a more interesting and instructive pursuit than the examination of the workshops of their own country, which contain within them a rich mine of knowledge, too generally neglected by the wealthier classes.


It has been my endeavour, as much as possible, to avoid all technical terms, and to describe, in concise language, the arts I have had occasion to discuss. In touching on the more abstract principles of political economy, after shortly stating the reasons on which they are founded, I have endeavoured to support them by facts and anecdotes; so that whilst young persons might be amused and instructed by the illustrations, those of more advanced judgment may find subject for meditation in the general conclusions to which they point. I was anxious to support the principles which I have advocated by the observations of others, and in this respect I found myself peculiarly fortunate. The Reports of Committees of the House of Commons, upon various branches of commerce and manufactures, and the evidence which they have at different periods published on those subjects, teem with information of the most important kind, rendered doubly valuable by the circumstances under which it has been collected. From these sources I have freely taken, and I have derived some additional confidence from the support they have afforded to my views.*


CHARLES BABBAGE.


DORSET STREET, MANCHESTER SQUARE, 


June 8, 1832.


* I am happy to avail myself of this occasion of expressing my obligations to the Right Hon. Manners Sutton, the Speaker of the House of Commons, to whom I am indebted for copies of a considerable collection of those reports.

\end{document}
